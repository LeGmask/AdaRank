\LoadClass{NewTeX}
\documentclass{NewTeX}
\usepackage[utf8]{inputenc}
\usepackage{tabularx}
\title{Projet PageRank \\ Script de démonstration}
\author{\'Equipe  EF7 : Evann DREUMONT  et  Timothée KLEIN}
\date{Décembre 2023}

\begin{document}

\maketitle

\section{Benchmark}

Lors de la démonstration, nous allons tout d'abord lancer le script de benchmark pour montrer que le script fonctionne pour chacun des fichiers networks :  

\begin{code}{sh}
./benchmark.sh
\end{code}

\section{Tests de la robustess}

En parallèle du benchmark, nous allons démontrer de la robustesse de notre programme, nous allons exécuter différentes commandes ne fonctionnant pas, ou en prenant des fichiers n'étant pas conforme.

\subsection{Commandes invalides}

\begin{code}{sh}
make run ARGS="-K aaaaaaa test.net"
\end{code}

\begin{code}{sh}
make run ARGS="-A aaaaaaa test.net"
\end{code}

\begin{code}{sh}
make run ARGS="-E aaaaaaa test.net"
\end{code}

\begin{code}{sh}
make run ARGS="-K -1 test.net"
\end{code}

\begin{code}{sh}
make run ARGS="-A 2 test.net"
\end{code}

\begin{code}{sh}
make run ARGS="-E -8 test.net"
\end{code}

\begin{code}{sh}
make run ARGS="-LOLOL -8 test.net"
\end{code}

\subsection{Fichier invalides}

\begin{code}{sh}
make run ARGS="jenexistepas.net"
\end{code}

Nous allons changer le fichier test.net pour ajouter un lien qui implique un nœud qui n'existe pas.

\begin{code}{sh}
make run ARGS="networks/test.net"
\end{code}


\section{Démonstration du bon fonctionnement de notre programme}

Pour cette partie, nous utiliserons le script python qui nous permet de faire une visualisation :

\begin{code}{sh}
cd vizualizer && poetry shell && cd ../
python vizualizer/main.py
\end{code}

Nous allons à présent montrer que notre programme est bien fonctionnel, en rajoutant tout d'abord une page qui référence plein de page, puis une page très référencée.

Pour finir, nous allons montrer que le programme en mode plein et creux fournis des résultats similaires.
\begin{code}{sh}
make run ARGS="networks/sujet.net"
make run ARGS="-C networks/sujet.net"
\end{code}
\end{document}